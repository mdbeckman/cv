
%%%%%%%%%%%%%%%%%%%%%%%%%%%%%%%%%%%%%%%%%
% Medium Length Professional CV
% LaTeX Template
% Version 2.0 (8/5/13)
%
% This template has been downloaded from:
% http://www.LaTeXTemplates.com
%
% Original author:
% Trey Hunner (http://www.treyhunner.com/)
%
% Important note:
% This template requires the resume.cls file to be in the same directory as the
% .tex file. The resume.cls file provides the resume style used for structuring the
% document.
%
%%%%%%%%%%%%%%%%%%%%%%%%%%%%%%%%%%%%%%%%%

%----------------------------------------------------------------------------------------
%	PACKAGES AND OTHER DOCUMENT CONFIGURATIONS
%----------------------------------------------------------------------------------------

\documentclass{resume} % Use the custom resume.cls style

\usepackage[left=0.75in,top=0.6in,right=0.75in,bottom=0.6in]{geometry} % Document margins
\usepackage{url}



\name{Matthew D. Beckman} % Your name
\address{464 Glenwood Avenue, Saint Paul, MN 55113} % Your address
\address{612.655.5235 \\ beckm109@umn.edu} % Your phone number and email

\begin{document}

%----------------------------------------------------------------------------------------
%	EDUCATION SECTION
%----------------------------------------------------------------------------------------

\begin{rSection}{Education}

\begin{rSubsection}{Ph.D. Statistics Education}{expected September 2015}{University of Minnesota}{Minneapolis, MN}
	\item[] Dissertation Title: Assessment of Cognitive Transfer Outcomes of Introductory Statistics Students.
	\item[] Co-Advisors: Dr. Joan Garfield \& Dr. Robert C. delMas 
\end{rSubsection}

\begin{rSubsection}{M.Sc. Statistics}{2008}{University of Minnesota}{Minneapolis, MN}
	\item[] Masters Project: Equivalence testing in quality control.
	\item[] Advisor: Dr. Douglas M. Hawkins 
\end{rSubsection}

\begin{rSubsection}{B.Sc. Mathematics}{2006}{Pennsylvania State University}{University Park, PA}
	\item[] Mathematics major with secondary education concentration.
\end{rSubsection}

\textbf{Pennsylvania Secondary Mathematics Teaching Certification} \hfill 2006
\vspace{0.5em}

\end{rSection}

%----------------------------------------------------------------------------------------
%	INDUSTRY EXPERIENCE SECTION
%----------------------------------------------------------------------------------------

\begin{rSection}{Industry Experience}

\begin{rSubsection}{Senior Statistician}{2008 - 2013; 2014 - Present}{Medtronic, Inc\textemdash  Neuromodulation}{Minneapolis, MN}
\item[] Provide eclectic analyses of post-market quality, reliability, medical safety, and pre-clinical, manufacturing, microbiology, research \& development data.  Also, co-teach various full-day professional development workshops on a regular basis.
\end{rSubsection}

%------------------------------------------------

\begin{rSubsection}{Senior Biostatistician}{2013 - 2014}{Nonin Medical, Inc}{Minneapolis, MN}
	\item[] Responsible for statistical leadership in all areas of the company, including design and analysis of clinical, research \& development, and manufacturing studies. 
\end{rSubsection}

%------------------------------------------------

\begin{rSubsection}{Independent Statistical Consulatant}{2010 - Present}{}{}
	\item[] Recent clients include legal, small business, and education research sectors.
\end{rSubsection}

%------------------------------------------------

\begin{rSubsection}{Statistics Internship}{2007 - 2008}{Ecolab, Inc\textemdash Research, Development \& Engineering}{Eagan, MN}
	\item[] Taught and developed training materials for various full-day statistics classes. Provided statistical consulting services to Ecolab colleagues nation-wide.
\end{rSubsection}


\end{rSection}

\clearpage
%----------------------------------------------------------------------------------------
%	TEACHING EXPERIENCE
%----------------------------------------------------------------------------------------

\begin{rSection}{Teaching Experience}

	\begin{rSubsection}{Department of Educational Psychology\textemdash University of Minnesota}{Minneapolis, MN}{Instructor}{}
		\item[] \textbf{EPSY 5261\textemdash Introductory Statistical Methods} \hfill \textit{Fall 2014}
		\item[] Activity-based course using flipped classroom approach and Lock \textit{et. al} (2013) \textit{Statistics: Unlocking the power of data} materials. Introduction to statistical reasoning for graduate students in non-quantitative disciplines and advanced undergraduate students.  Similar audience to previous experience with this course several years earlier; innovative new curriculum with 50\% simulation-based methods and 50\% conventional (i.e. non-simulation) methods. \vspace{0.5em}
		\item[] \textbf{EPSY 5261\textemdash Introductory Statistical Methods} \hfill \textit{Fall 2008; Spring 2009}
		\item[] Introduction to statistical reasoning for graduate students in non-quantitative disciplines and advanced undergraduate students.  Activity-based curriculum based on conventional (i.e. non-simulation) methods. 
		
	\end{rSubsection}

	%------------------------------------------------
	
	\begin{rSubsection}{Department of Mathematics \& Computer Science\textemdash Saint Olaf College}{Northfield, MN}{Instructor}{}
		\item[] \textbf{STAT 212\textemdash Statistics for the Sciences} \hfill \textit{Spring 2015}
		\item[] Activity-based course using Chance \& Rossman's \textit{Investigating Statistical Concepts, Applications, and Methods} materials. Topics include probability models, exploratory graphics, descriptive techniques, statistical designs, hypothesis testing, confidence intervals, and simple/multiple regression for undergraduate students in quantitative and scientific concentrations.
	\end{rSubsection}
	
%------------------------------------------------
	
	\begin{rSubsection}{Department of Statistics\textemdash University of Minnesota}{Minneapolis, MN}{Teaching Assistant}{}
		\item[] \textbf{STAT 3011\textemdash Introduction to Statistical Analysis} \hfill \textit{Fall 2007; Spring 2008}
		\item[] Introduction to statistical reasoning for second- and third-year undergraduate students in physical and social science majors. Responsibilities included weekly office hours, lab teaching, and grading. \vspace{0.5em}
		\item[] \textbf{STAT 1001\textemdash Introduction to Ideas of Statistics} \hfill \textit{Fall 2006; Spring 2007}
		\item[] Statistical literacy course for first-year undergraduate students. Responsibilities included weekly office hours, lab teaching, and grading.
		
	\end{rSubsection}


	
	%------------------------------------------------
		
	\begin{rSubsection}{Medtronic, Inc}{Minneapolis, MN}{Professional Development Instructor}{2008 - Present}
		\item[] Preclinical Study Design
		\item[] Design and Analysis of Experiments
		\item[] Statistical Methods for Engineers
		
	\end{rSubsection}
	
	%------------------------------------------------
	
	\begin{rSubsection}{Ecolab, Inc}{Eagan, MN}{Professional Development Instructor}{2007 - 2008}
		\item[] Statistical Process Control
		\item[] Measurement Systems Analysis
		\item[] Design of Experiments
		\item[] Robust Design and Specification Analysis
%		\item[] Basic Data Analysis and Graphs		
		
	\end{rSubsection}
	
	
	%------------------------------------------------
	
	\begin{rSubsection}{Corry Area High School}{Corry, PA}{Algebra Teacher}{Summer 2006}
		\item[] Prepared and executed original lessons and assessments for a class of 16 students including Sophomores, Juniors, and Seniors that had previously failed Algebra I one or more times.
	\end{rSubsection}
	
		
	%------------------------------------------------

	
	\begin{rSubsection}{General McLane High School}{Edinboro, PA}{Mathematics Tutor}{Summer 2006}
		\item[] Tutored students referred to me by the school district.
	\end{rSubsection}
	
	%------------------------------------------------
	
\end{rSection}


%----------------------------------------------------------------------------------------
%	PUBLICATIONS
%----------------------------------------------------------------------------------------


\begin{rSection}{Publications}

\textbf{Beckman, M. D.}, delMas, R. C., and Garfield, J. (in review). Cognitive transfer outcomes for a simulation-based introductory statistics curriculum. \textit{Statistics Education Research Journal}.\vspace{0.5em}

Singh, H.,  \textbf{Beckman, M.},  Brown, K., Beebe, D., Adhikari, R., Belani, K. G. Comparison of normal regional Oxygen saturation readings and repeatability across diverse patient populations. \textit{Submission Pending to Anesthesia and Analgesia}. %\vspace{-.2em}
	\begin{rSubsection}{}{}{}{}
		\item[]  This paper is the result of collaboration between the University of Minnesota and Nonin Medical. My contributions included primary authorship of the Discussion and Limitations sections in addition to conducting and interpreting the statistical analyses. 		
	\end{rSubsection}
	
\end{rSection}


%----------------------------------------------------------------------------------------
%	PRESENTATIONS
%----------------------------------------------------------------------------------------

\begin{rSection}{Colloquia}
	

	\begin{rSubsection}{Invited Presentation}{}{}{}
		
		\item[] \textbf{Beckman, M.} (2015). Teaching for Transfer in the Statistics Classroom. \textit{Twin Cities Stat Chat}. Saint Paul, MN. \vspace{0.5em}

		\item[] \textbf{Beckman, M.} (2015). Cognitive Transfer Outcomes for Introductory Statistics Students. \textit{Colloquium Sponsored by Penn State Department of Statistics}. University Park, PA. \vspace{0.5em}
		
		\item[] \textbf{Beckman, M.} (2015). Cognitive Transfer Outcomes for Introductory Statistics Students. \textit{Colloquium Sponsored by Cal Poly Department of Statistics}. San Luis Obispo, CA. \vspace{0.5em}		
		
		\item[] \textbf{Beckman, M.} (2015). Cognitive Transfer in the Introductory Statistics Curriculum. \textit{Twin Cities Stat Chat}. Saint Paul, MN. \vspace{0.5em}	

		\item[] \textbf{Beckman, M.}, Keenan, T., (2013). Critique of the Johnson family of transformations to Normality. \textit{Medtronic Statistics Conference}. Minneapolis, MN. \vspace{0.5em}
	
		\item[] \textbf{Beckman, M.} (2012). Design and analysis of experiments for pre-clinical research.  \textit{Medtronic Statistics Conference}. Minneapolis, MN. \vspace{0.5em}
		
		\item[] \textbf{Beckman, M.} (2008). Complaint trending for post-market surveillance. \textit{6th Annual Product Complaints Congress for Life Sciences}. Center for Business Intelligence. Washington, D.C. \vspace{0.5em}
		
		\item[] \textbf{Beckman, M.} (2008). Statistical analysis for corrective \& preventative action (CAPA) effectiveness. \textit{6th Annual Product Complaints Congress for Life Sciences}. Center for Business Intelligence. Washington, D.C. \vspace{0.5em}

			
	\end{rSubsection}	

%	\begin{rSubsection}{Contributed Panel}{}{}{}
		
%		\item[] \textbf{Beckman, M., Brown, E., Fry, E., Garfield, J., Sabbag, A., Ziegler, L.} (2015). The Quest for Good Assessments for Research and Evaluation. \textit{Joint Statistical Meetings}. American Statistical Association. Seattle, WA. \vspace{0.5em}
		
%	\end{rSubsection}

	
	\begin{rSubsection}{Contributed Poster}{}{}{}
		
		\item[] Wu, J., Lai, C., \textbf{Beckman, M.}, Raike, R., Gupta, R., Abosch, A., Nelson, D. (2013). Video-motion detection for objectively quantifying movements in patients with Parkinson's disease. \textit{17th International Congress of Parkinson's Disease and Movement Disorders.} Movement Disorder Society. Sydney, Australia. \vspace{0.5em}
		
		\item[] Doe, B., Fontecchio, J., \textbf{Beckman, M.}, Depre, J., Boulware, S., Keenan, T. (2012). Visual management system for manufacturing yield SPC data. \textit{Medtronic Science and Technology Conference}. Minneapolis, MN. \vspace{0.5em}
		
		\item[] Holland, M., \textbf{Beckman, M.} (2011). Statistical methods for monitoring rare adverse events. \textit{Neuromodulation Innovation Week}. Minneapolis, MN.   \vspace{0.5em}
		
	\end{rSubsection}

	\begin{rSubsection}{Miscellaneous}{}{}{}
		
		\item[] Beckman, M., Brown, E., delMas, R., Fry, E., Justice, N., Sabbag, A. (2014, November 21). Simulation-based statistical inference: Different tools for different audiences [web log post]. Retrieved from \url{https://www.causeweb.org/sbi/?p=422#more-422} \vspace{0.5em}
		
		
	\end{rSubsection}
	
	
	
\end{rSection}


%----------------------------------------------------------------------------------------
%	RESEARCH EXPERIENCE
%----------------------------------------------------------------------------------------


\begin{rSection}{Research Experience}
	
	
	\begin{rSubsection}{Models of Statistical Thinking (MOST) Assessment}{Spring 2013}{}{}
		\item[] Conducted interviews with expert reviewers in order to gather validity evidence during development of the MOST assessment as part of the CATALST Project (NSF DUE-0814433) headed by Dr. Joan Garfield (University of Minnesota, Twin Cities).
		
	\end{rSubsection}
	
	
\end{rSection}


%----------------------------------------------------------------------------------------
%	EVALUATION EXPERIENCE
%----------------------------------------------------------------------------------------

\begin{rSection}{Evaluation Experience}
	
	
	\begin{rSubsection}{Introductory Statistics Redesign Materials Critique}{Spring 2011}{}{}
		\item[] Member of the evaluation team for the Creating a Teaching and Learning Infrastructure for Introductory Statistics Redesign Project (NSF DUE-0737126) headed by Dr. Robert Gould (University of California, Los Angeles). 
		
	\end{rSubsection}
	
	
\end{rSection}

%----------------------------------------------------------------------------------------
%	SERVICE
%----------------------------------------------------------------------------------------

\begin{rSection}{Service}
	
	
	\begin{rSubsection}{Reviewer}{}{}{}
		\item[] \textit{Technology Innovations in Statistics Education} (Spring 2015 - present)
		\item[] \textit{Journal of Statistics Education} (Spring 2015 - present)

	\end{rSubsection}
	
	
\end{rSection}


%----------------------------------------------------------------------------------------
%	PROFESSIONAL DEVELOPMENT
%----------------------------------------------------------------------------------------

\begin{rSection}{Professional Development}
	
	
	\begin{rSubsection}{United States Conference on Teaching Statistics (USCOTS)}{Spring 2013}{Raleigh-Durham, NC}{}
		\item[] Member of the evaluation team for the Creating a Teaching and Learning Infrastructure for Introductory Statistics Redesign Project (NSF DUE-0737126) headed by Dr. Robert Gould (University of California, Los Angeles). 
		
	\end{rSubsection}
	
	
	\begin{rSubsection}{Twin Cities Stat Chat}{2009 - Present}{Macalester College, Saint Paul MN}{}
		\item[] Regular attendee of periodic meetings among Statistics instructors and researchers representing a variety of colleges and Universities in the greater Twin Cities area.  Content includes research seminars, guest speakers, article discussion, and teaching materials.
		
		
	\end{rSubsection}
	
		
	\begin{rSubsection}{Medtronic Statistics Conference}{2009 - 2014}{Medtronic World Headquarters, Minneapolis MN}{}
		\item[] Attended presentations showcasing the work of other Medtronic Statisticians worldwide, as well as half-day and full-day professional development topics. \vspace{0.5em}  
		\item[] \textit{Enhancing Big Data Projects through Statistical Engineering} half-day seminar presented by Ronald D. Snee (Snee Associates, LLC). 2014. \vspace{0.5em}  
		\item[] \textit{Statistical Design of Sequential Clinical Trials in R} full-day seminar presented by Scott S. Emerson (University of Washington). 2013.\vspace{0.5em}  
		\item[] \textit{Variation in Decomposition} half-day seminar and \textit{Regulatory Trends} half-day seminar presented by Wayne Taylor (Taylor Enterprises, Inc). 2013.\vspace{0.5em}  
		\item[] \textit{Propensity Score Matching} full-day seminar presented by Thomas E. Love (Case Western Reserve University). 2012.\vspace{0.5em}  
		\item[] \textit{Experiments for Robust Design} full-day seminar presented by Connie M. Borror (Arizona State University). 2012.\vspace{0.5em}  
		\item[] \textit{Bayesian Adaptive Methods for Clinical Trials} full-day seminar presented by Bradley P. Carlin (University of Minnesota) and Andrew Mugglin (Paradigm Biostatistics, LLC). 2011.\vspace{0.5em}  
		\item[] \textit{Statistical Methods for Reliability Data} full-day seminar presented by William Q. Meeker (Iowa State University). 2010.\vspace{0.5em}  
		\item[] \textit{Statistical Process Control} full-day seminar by Wayne Taylor (Taylor Enterprises, Inc). 2009.\vspace{0.5em}  
		
		
	\end{rSubsection}
		
		
\end{rSection}



%----------------------------------------------------------------------------------------
%	AWARDS AND HONORS SECTION
%----------------------------------------------------------------------------------------

\begin{rSection}{Honors \& Accomplishments}
	
	\begin{tabular}{ @{} >{\bfseries}r @{\hspace{6ex}} l }
		2014 \\ Nonin Medical & \begin{minipage}[t]{0.75\columnwidth} Nominated for \textit{NONIN WINS} employee recognition award for outstanding contribution to publishable research with physician partners at the University of Minnesota. \end{minipage}\tabularnewline \\
		
		2013 \& 2009 \\ Medtronic  & \begin{minipage}[t]{0.75\columnwidth} Presented with a long-term incentive award in 2009 and again in 2013 designed to recognize and retain high-achieving employees. \end{minipage}\tabularnewline \\
		
		2006 \\ Penn State & \begin{minipage}[t]{0.75\columnwidth} Graduated with “High Distinction” for achieving class rank in top 4\% of peers. \\ Selected as one of 18 inductees to Penn State S\&B Senior Honor Society.\end{minipage}\tabularnewline \\
		
		2005 \\ Penn State & \begin{minipage}[t]{0.75\columnwidth} Recognized by Student-Athlete Advisory Board and \textit{Penn Stater} Magazine for achieving the highest cumulative GPA of any active varsity athlete\end{minipage}\tabularnewline \\
		
		Miscellanea & \begin{minipage}[t]{0.75\columnwidth} Twin Cities marathon finisher; personal finance instructor; NCAA division I pole vaulter and sprinter; downhill ski racer \& instructor; jazz, concert, and marching percussionist; Eagle Scout award recipient \end{minipage}\tabularnewline \\
	\end{tabular}
	
\end{rSection}


%----------------------------------------------------------------------------------------
%	EXAMPLE SECTION
%----------------------------------------------------------------------------------------

%\begin{rSection}{Section Name}

%Section content\ldots

%\end{rSection}

%----------------------------------------------------------------------------------------

\end{document}
